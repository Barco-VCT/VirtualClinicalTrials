\usepackage[obeyspaces]{url}
\usepackage{amsmath}
\usepackage{babel}
\usepackage{color}
\usepackage[english,ruled,vlined,longend]{algorithm2e}
\usepackage{epsfig} % for the figures and the pnd/jpg images
\usepackage{fancyhdr}
\usepackage[a4paper,twoside,textwidth=17cm,top=2.2cm,bottom=2.2cm]{geometry} % for the page geometry 
\usepackage[a4paper,plainpages=false]{hyperref} % for the hyperlinks
\usepackage[latin1]{inputenc} % for the accents: � � � �
\usepackage{listings}
\usepackage{minitoc} % for the short contents for each chapter
\usepackage{shorttoc} % for the short contents
\usepackage{times}
\usepackage{verbatim}
\usepackage[toc,acronym]{glossaries} % for glossaries and acronyms
\usepackage{colortbl}
\usepackage{subfig}
\usepackage{amsfonts}
\usepackage{stmaryrd}
\usepackage{tikz}
\usepackage{fancybox}
\usepackage{relsize}
\usepackage{pifont}
\usepackage{calc}
\usepackage{fix-cm}
\usepackage{mdwlist}
\usepackage{enumerate}
\usepackage{enumitem}
\usepackage{floatflt}
\usepackage{amssymb}
\usepackage{mathrsfs}
\usepackage[all]{xy}
\usepackage{ulem}
\usepackage{theorem}
\usepackage{thmbox}


%%%%%%%%%%%%%%%%%%%%%%%%%%%%%%%%%%%%%%%%%%
% To insert code in tex files
% http://en.wikibooks.org/wiki/LaTeX/Packages/Listings
% use command: 
% \lstset{ %
% language=C++ % or Matlab or XML...
% }
% \lstinputlisting{./appendix1b/TemporalResponseModule.cpp}
%%%%%%%%%%%%%%%%%%%%%%%%%%%%%%%%%%%%%%%%%%
\lstset{ %
keywordstyle=\bf \color{blue},	% choose the color of the language's keywords
commentstyle=\color[rgb]{0.0,0.45,0.0},	% choose the color of the comments in the code
stringstyle=\color{red},				% choose the color of the strings in the code
identifierstyle=\bf,						% choose the style of the variable's identifiers
basicstyle=\footnotesize,       % the size of the fonts that are used for the code
backgroundcolor=\color{white},  % choose the background color. You must add \usepackage{color}
showspaces=false,               % show spaces adding particular underscores
showstringspaces=false,         % underline spaces within strings
showtabs=false,                 % show tabs within strings adding particular underscores
frame=none,			% adds a frame around the code
tabsize=2,			% sets default tabsize to 2 spaces
captionpos=b,			% sets the caption-position to bottom
breaklines=true,		% sets automatic line breaking
breakatwhitespace=false,	% sets if automatic breaks should only happen at whitespace
escapeinside={\%*}{*)}          % if you want to add a comment within your code
}  

\makeatletter
%SECTION
\renewcommand{\thesection}{\Roman{section}}
%SUBSECTION
\renewcommand{\thesubsection}{\arabic{subsection}}
%SUBSUBSECTION
\renewcommand{\thesubsubsection}{\alph{subsubsection}}
%integers
\newcommand{\integer}[2]{\llbracket #1,#2 \rrbracket}
%scalar product
\newcommand{\scalar}[2]{#1\cdot#2}
%norm
\newcommand{\norm}[1]{\left\|#1\right\|}
%functionPage
\newcommand{\functionPagePage}[5]{\begin{array}{lrcl}
#1: & #2 & \longrightarrow & #3 \\
    & #4 & \longmapsto & #5 \end{array}}
\newcommand{\vectorPage}[3]{\left(\begin{array}{l}
#1\\
#2\\
#3\\
\end{array}\right)}
%Matrice 3,3
\newcommand{\matrixcoeff}[9]{
\left(
\begin{array}{ccc}
#1 & #4 & #7\\
#2 & #5 & #8\\
#3 & #6 & #9
\end{array}
\right)
}
\newcommand{\matrixvector}[3]{
\left(
\begin{array}{ccc}
#1 & #2 & #3
\end{array}
\right)
}
\makeatother
\theoremstyle{break}
%Definition
\newtheorem[L]{defi}{Definition}
%Theorem
\newtheorem[L]{theo}{Theorem}
%Proposal
\newtheorem[L]{prop}{Proposal}
%Lemma
\newtheorem[L]{lemma}{Lemma}
%Corrolar 
\newtheorem[L]{cor}{Corollary}[theo]
%Application
\newtheorem[L]{app}{Application}[theo]
%Formula
\newtheorem[L]{formula}{Formula}