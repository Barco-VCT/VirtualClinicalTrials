\chapter{Clinical study simulation}
\minitoc{}

The goal of this simulation is to run a virtual clinical trial with breast images that were previously used in the study described in \cite{Mar12}. The breast images were generated using the Bakic's anthropomorphic breast phantom developed at the University of Pennsylvania.

In this virtual clinical trials 8 bits breast images with and without signals are used as input. Half of the images do contain a signal in the center of the images and half do not contain signals.

These images are then used for simulating the displayed images represented in the $XYZ$ color space. The simulated is a grayscale mammographic display with a contrast of $1200:1$ with a maximum luminance of $600cd/m^2$. The full chain is in 8 bits.

Then the second component $Y$ is used by a single slice CHO with Laguerre-Gauss channels \cite{Pla09a,Mye87}.

Finally a Multi-reader-Multi-Cases study is carried out by training and testing multiple single slice CHO observers using the technique described by Gallad in \cite{Gal06}.


\section{Introduction}

There are three steps for this simulation:

\begin{itemize}
\item Simulate displayed images
\item Simulate one observer
\item Run statistical analysis
\end{itemize}

\section{Simulate displayed images or "00\_displaySimulation"}

\subsection{Input image}

The input image is a 8 bit image saved in a raw file (64x64pixels stored as 16 bits unsigned) (cf Figure \ref{fig:breast}). Even stored in 16 bits unsigned, the pixel values are all between "0" and "1".

To open it with ImageJ by instance, use "import raw file" with the following options:
\begin{itemize}
\item Image type: 16 bit unsigned RGB
\item Width: 64 pixels
\item Height: 64 pixels
\item Offset to first image: 0 bytes
\item Number of images: 1
\item Gap between images: 0
\item White is not zero
\item Little-endian byte order
\item Not open all files in folder
\item Not use virtual stack
\end{itemize}

\begin{figure}[!htb]
\begin{center}
\includegraphics[width=0.25\columnwidth]{./05_ClinicalStudy/images/breast.png}
\caption{Breast image.}\label{fig:breast}
\end{center}
\end{figure}

\subsection{XML file}

This xml is used for generating the outcome of a monitor on which breast images are displayed.The modules except "DisplayLutModule" "ConversionDDL2CDModule" have been described in the previous chapters.

\lstset{language=XML}
\begin{lstlisting}
<?xml version="1.0" encoding="ISO-8859-1"?>
<SuperXML>
	<LIST_OF_ITERATIONS name = "pipeline" value="1">	
	</LIST_OF_ITERATIONS>
	
	<LIST_OF_PARAMETERS>
		<LocalParameter name = "REF">
			<SequenceRawGeneratorModule name="directory" value = "#0"/>
			<WriterModule name="Filename" value="#2"/>
		</LocalParameter>				
	</LIST_OF_PARAMETERS>

	<TEMPLATE_MEVIC_SIMULATION>
		<REF>
			<SequenceRawGeneratorModule>
				<directory dataType="char" value = "$"/>
				<number_of_slices dataType="int" value = "1"/>
				<width dataType = "int" value = "64"/>
				<height dataType = "int" value = "64"/>
				<nbBitsRange dataType = "int" value = "8"/>
				<_0bigEndian_1littleEndian dataType = "int" value = "1"/>
				<nbBitsOutput dataType = "int" value = "8"/>
				<nbBitsPrecision dataType = "int" value = "16"/>
				<_1WhiteIs0_0Otherwise dataType = "int" value = "0"/>
				<frame_repeat dataType = "int" value = "1"/>
				<_1RGB_0GRAY dataType = "int" value = "0"/>			
			</SequenceRawGeneratorModule>
			 <VideoCardModule name="VideoCardModule">
				<number_of_bits_out dataType="int" description="LUT" value="8" />
			</VideoCardModule>
		 	<DisplayModule>
				<_1Color_0Monochrome dataType="int" value="0"/>
				<filenameNativeCurveGray dataType="char" value = ".\input\gray.csv"/>
				<nbBits dataType="int" value="8"/>
				<frequency dataType="int" value="50"/>	
			</DisplayModule>
			<DisplayLutModule>
			</DisplayLutModule>
			<ConversionDDL2CDModule name ="ConversionDDL2CDModule">
				<_1ddl2cd_0cd2ddl dataType="char" value="1"/>
			</ConversionDDL2CDModule>
			<SaveFrameTXTModule>
				<Filename dataType="char" value="#1\GRAYIN"/>
				<ComponentToWrite dataType="int" value="0"/>
				<FrameToWrite dataType="int" value="-1"/>
				<ChannelToWrite dataType="int" value="-1"/>
			</SaveFrameTXTModule>
			<SaveFrameTXTModule>
				<Filename dataType="char" value="#1\VIDEOCARD"/>
				<ComponentToWrite dataType="int" value="1"/>
				<FrameToWrite dataType="int" value="-1"/>
				<ChannelToWrite dataType="int" value="-1"/>
			</SaveFrameTXTModule>
			<SaveFrameTXTModule>
				<Filename dataType="char" value="#1\GRAYGSDF"/>
				<ComponentToWrite dataType="int" value="2"/>
				<FrameToWrite dataType="int" value="-1"/>
				<ChannelToWrite dataType="int" value="-1"/>
			</SaveFrameTXTModule>	
			<SaveFrameTXTModule>
				<Filename dataType="char" value="#1\XYZ"/>
				<ComponentToWrite dataType="int" value="3"/>
				<FrameToWrite dataType="int" value="-1"/>
				<ChannelToWrite dataType="int" value="-1"/>
			</SaveFrameTXTModule>
			<WriterModule>
				<Filename dataType="char" value="$"/>
				<compression dataType="bool" value="1"/>
				<savelastcomponent dataType="bool" value="1"/>
			</WriterModule>	
		</REF>	
	</TEMPLATE_MEVIC_SIMULATION>
</SuperXML>
\end{lstlisting}

\subsection{Command}

\lstset{language=C++}
\begin{lstlisting}
run_simulation.bat
\end{lstlisting}

In this command the VCT.exe is executed several times for generating the necessary input for the virtual observer displayed images with and without signals. The images with signals are named "diseased" and the ones without signals are named "healthy".

Part of the file "run\_simulation.bat":

\lstset{language=C++}
\begin{lstlisting}
@echo off
"../../vct/VCT.exe" SuperXML "./xml/test_DisplayLutModule.xml" "./input/diseased/train/"0000 "./output/diseased/train/"0000 "./output/diseased/train/0000.bin"
...
...
...
"../../vct/VCT.exe" SuperXML "./xml/test_DisplayLutModule.xml" "./input/healthy/test/"0049 "./output/healthy/test/"0049 "./output/healthy/test/0049.bin"
\end{lstlisting}

Then the data are generated and organized correctly for running the single slice CHO.

\subsection{Module: DisplayLutModule}

This module does not take any input parameter. It applies a 1D LUT to the RGB digital driving levels for calibrating the display according to the Grayscale Standard Display Function as described in \cite{DIC98}. Based on the native curve given as input to the module DisplayModule and based on the minimum and maximum luminance values, the 1D LUT can be generated and applied.

After this step, the RGB levels are transfomed in such way that the perceived differences between two consecutive grayscale levels are constant over the full range. The perception of grayscale levels has been linearized.

\subsection{Module: ConversionDDL2CDModule}

This module converts the RGB into XYZ values using the input native curve.

The outcome is 2D maps with $X$, $Y$ and $Z$ values.

The simulation of the display generates .txt and .raw that can be easily opened with ImageJ by instance.

The simulation generates also 

\section{Simulate one single slice observer: "01\_ssCHOSimulation"}

The files generated by the display simulation are used by this module for running a clinical study. An observer is trained with 100 images and then tested with 100 other images.

\subsection{XML file}

This xml is used for generating the outcome of a monitor on which breast images are displayed.

\lstset{language=XML}
\begin{lstlisting}
<?xml version="1.0" encoding="ISO-8859-1"?>
<SuperXML>
	<LIST_OF_ITERATIONS name = "pipeline" value="1">	
	</LIST_OF_ITERATIONS>
	
	<LIST_OF_PARAMETERS>
	</LIST_OF_PARAMETERS>

	<TEMPLATE_MEVIC_SIMULATION>
		<REF>
			<SingleSliceCHOModule>
				<reader_id dataType = "unsigned long" value = "1"/>
				<image_height dataType = "unsigned long" value = "64"/>
				<image_width dataType = "unsigned long" value = "64"/>
				<image_depth dataType = "unsigned long" value = "1"/>
				<one_image_file_per_frame dataType = "bool" value = "false"/>
				<target_slice_id dataType = "unsigned long" value = "1"/>
				<num_image_pairs_training dataType = "unsigned long" value = "50"/>
				<num_image_pairs_testing dataType = "unsigned long" value = "50"/>
				<channels_number dataType = "unsigned long" value = "15"/>	
				<channels_type dataType = "char" value = "LG"/>
				<channels_lg_spread dataType = "float" value = "22"/>
				<save_channelized_images dataType = "bool" value = "false"/>
				<auto_create_outputfolders dataType = "bool" value = "true"/>
				<dirpath_in_training_healthy dataType = "char" value = "#1\00_displaySimulation\output\healthy\train"/>
				<dirpath_in_training_diseased dataType = "char" value = "#1\00_displaySimulation\output\diseased\train"/>
				<dirpath_in_testing_healthy dataType = "char" value = "#1\00_displaySimulation\output\healthy\test"/>
				<dirpath_in_testing_diseased dataType = "char"  value = "#1\00_displaySimulation\output\diseased\test"/>
				<dirpath_out_channelized_training_healthy dataType = "char" value = "#0"/>
				<dirpath_out_channelized_training_diseased dataType = "char" value = "#0"/>
				<dirpath_out_channelized_testing_healthy dataType = "char" value = "#0"/>
				<dirpath_out_channelized_testing_diseased dataType = "char" value = "#0"/>
				<filepath_out_training_tpf_fpf dataType = "char" value = "#0\outputTest\training_tpf_fpf.txt"/>
				<filepath_out_training_auc_snr dataType = "char" value = "#0\outputTest\training_auc_snr.txt"/>
				<filepath_out_testing_tpf_fpf dataType = "char" value = "#0\outputTest\testing_tpf_fpf.txt"/>
				<filepath_out_testing_auc_snr dataType = "char" value = "#0\outputTest\testing_auc_snr.txt"/>
			 </SingleSliceCHOModule>
		</REF>	
	</TEMPLATE_MEVIC_SIMULATION>
</SuperXML>
\end{lstlisting}

\subsection{Command}

\lstset{language=C++}
\begin{lstlisting}
run_simulation.bat
\end{lstlisting}

or 

\lstset{language=C++}
\begin{lstlisting}
"..\..\vct\VCT.exe" SuperXML ".\xml\test_ssCHO.xml" "." "../"
\end{lstlisting}

In this command the VCT.exe is executed for training and testing an observer.

\subsection{output}

Four files are generated:

\begin{itemize}
\item training\_tpf\_fpf.txt: True Positive Fraction in function of False Positive Fraction, training phase
\item training\_auc\_snr.txt: AUC (Area Under the Curve) and SNR (Signal Noise Ratio) SNR, training phase
\item testing\_tpf\_fpf.txt: True Positive Fraction in function of False Positive Fraction, testing phase
\item testing\_auc\_snr.txt: AUC (Area Under the Curve) and SNR (Signal Noise Ratio) SNR, testing phase
\end{itemize}

\subsection{SingleSliceCHOModule}

\subsubsection{Parameter: "reader\_id"}

This is the reader identification number.

\subsubsection{Parameter: "image\_height"}

This is the image width in pixels.

\subsubsection{Parameter: "image\_width"}

This is the image height in pixels.

\subsubsection{Parameter: "image\_depth"}

This is the image depths in frame number.

\subsubsection{Parameter: "one\_image\_file\_per\_frame"}

This is for indicating if the frames are stored within one file or split.  It must be equal to false for the single slice CHO.

\subsubsection{Parameter: "target\_slice\_id"}

This is the target slide or frame for training the observer. It must be equal to 1 for the single slice CHO.

\subsubsection{Parameter: "num\_image\_pairs\_training"}

This is the number of pairs of images for the training phase.

\subsubsection{Parameter: "num\_image\_pairs\_testing"}

This is the number of pairs of images for the testing phase.

\subsubsection{Parameter: "channels\_number"}

This is the number of LG channels.

\subsubsection{Parameter: "channels\_type"}

This is the type of channels but only Laguerre Gauss channels are supported.

\subsubsection{Parameter: "channels\_lg\_spread"}

This is a parameter intrinsic to LG channels, it the is the spread in pixels.

\subsubsection{Parameter: "save\_channelized\_images"}

This is for saving or not the channels.

\subsubsection{Parameter: "auto\_create\_outputfolders"}

This for automatically or not automatically creating the subfolders.

\subsubsection{Parameter: "dirpath\_in\_training\_healthy"}

This is the path of the training healthy images.

\subsubsection{Parameter: "dirpath\_in\_training\_diseased"}

This is the path of the training diseased images.

\subsubsection{Parameter: "dirpath\_in\_testing\_healthy"}

This is the path of the testing healthy images.

\subsubsection{Parameter: "dirpath\_in\_testing\_diseased"}

This is the path of the testing diseased images.

\subsubsection{Parameter: "dirpath\_out\_channelized\_training\_healthy"}

This is the output path of the training healthy results.

\subsubsection{Parameter: "dirpath\_out\_channelized\_training\_diseased"}

This is the output path of the training diseased results.


\subsubsection{Parameter: "dirpath\_out\_channelized\_testing\_healthy"}

This is the output path of the testing healthy results.


\subsubsection{Parameter: "dirpath\_out\_channelized\_testing\_diseased"}

This is the output path of the testing diseased results.


\subsubsection{Parameter: "filepath\_out\_training\_tpf\_fpf"}

This is the output path and filename for the training TPF vs FPF results for the training phase.

\subsubsection{Parameter: "filepath\_out\_training\_auc\_snr"}

This is the output path and filename for the AUC and SNR results for the training phase.

\subsubsection{Parameter: "filepath\_out\_testing\_tpf\_fpf"}

This is the output path and filename for the training TPF vs FPF results for the testing phase.

\subsubsection{Parameter: "filepath\_out\_testing\_auc\_snr"}

This is the output path and filename for the AUC and SNR results for the testing phase.


\section{Statistical analysis: "02\_MRMCSimulation"}

The files generated by the display simulation are used by this module for running a clinical study. 5 observers are trained and tested on the same dataset

\subsection{XML file}

This xml is used for generating the outcome of a monitor on which breast images are displayed. The modules except "MRMCModule" have been described in the previous chapters.

\lstset{language=XML}
\begin{lstlisting}
<?xml version="1.0" encoding="ISO-8859-1"?>
<SuperXML>
	<LIST_OF_ITERATIONS name = "pipeline" value="1">	
	</LIST_OF_ITERATIONS>
	
	<LIST_OF_PARAMETERS>
	</LIST_OF_PARAMETERS>

	<TEMPLATE_MEVIC_SIMULATION>
		<REF>
			<SingleSliceCHOModule>
				<reader_id dataType = "unsigned long" value = "1"/>
				<image_height dataType = "unsigned long" value = "64"/>
				<image_width dataType = "unsigned long" value = "64"/>
				<image_depth dataType = "unsigned long" value = "1"/>
				<one_image_file_per_frame dataType = "bool" value = "false"/>
				<target_slice_id dataType = "unsigned long" value = "1"/>
				<num_image_pairs_training dataType = "unsigned long" value = "16"/>
				<num_image_pairs_testing dataType = "unsigned long" value = "16"/>
				<channels_number dataType = "unsigned long" value = "15"/>	
				<channels_type dataType = "char" value = "LG"/>
				<channels_lg_spread dataType = "float" value = "22"/>
				<save_channelized_images dataType = "bool" value = "false"/>
				<auto_create_outputfolders dataType = "bool" value = "true"/>
				<dirpath_in_training_healthy dataType = "char" value = "#0\input\healthy\train1"/>
				<dirpath_in_training_diseased dataType = "char" value = "#0\input\diseased\train1"/>
				<dirpath_in_testing_healthy dataType = "char" value = "#0\input\healthy\test"/>
				<dirpath_in_testing_diseased dataType = "char"  value = "#0\input\diseased\test"/>
				<dirpath_out_channelized_training_healthy dataType = "char" value = ""/>
				<dirpath_out_channelized_training_diseased dataType = "char" value = ""/>
				<dirpath_out_channelized_testing_healthy dataType = "char" value = ""/>
				<dirpath_out_channelized_testing_diseased dataType = "char" value = ""/>
				<filepath_out_training_tpf_fpf dataType = "char" value = "#0\output_temp\tpf_fpf\training\reader_01_training_tpf_fpf.txt"/>
				<filepath_out_training_auc_snr dataType = "char" value = "#0\output_temp\auc_snr\training\reader_01_training_auc_snr.txt"/>
				<filepath_out_testing_tpf_fpf dataType = "char" value = "#0\output_temp\tpf_fpf\testing\reader_01_testing_tpf_fpf.txt"/>
				<filepath_out_testing_auc_snr dataType = "char" value = "#0\output_temp\auc_snr\testing\reader_01_testing_auc_snr.txt"/>
				<filepath_out_reader_score_healthy_ref dataType = "char" value = "#0\output_temp\scores\reader_00_scores_healthy.txt"/>
				<filepath_out_reader_score_healthy_new dataType = "char" value = "#0\output_temp\scores\reader_01_scores_healthy.txt"/>
				<filepath_out_reader_score_diseased_ref dataType = "char" value = "#0\output_temp\scores\reader_00_scores_diseased.txt"/>
				<filepath_out_reader_score_diseased_new dataType = "char" value = "#0\output_temp\scores\reader_01_scores_diseased.txt"/>
			 </SingleSliceCHOModule>
			 <SingleSliceCHOModule>
				<reader_id dataType = "unsigned long" value = "2"/>
				<image_height dataType = "unsigned long" value = "64"/>
				<image_width dataType = "unsigned long" value = "64"/>
				<image_depth dataType = "unsigned long" value = "1"/>
				<one_image_file_per_frame dataType = "bool" value = "false"/>
				<target_slice_id dataType = "unsigned long" value = "1"/>
				<num_image_pairs_training dataType = "unsigned long" value = "16"/>
				<num_image_pairs_testing dataType = "unsigned long" value = "16"/>
				<channels_number dataType = "unsigned long" value = "15"/>	
				<channels_type dataType = "char" value = "LG"/>
				<channels_lg_spread dataType = "float" value = "22"/>
				<save_channelized_images dataType = "bool" value = "false"/>
				<auto_create_outputfolders dataType = "bool" value = "true"/>
				<dirpath_in_training_healthy dataType = "char" value = "#0\input\healthy\train2"/>
				<dirpath_in_training_diseased dataType = "char" value = "#0\input\diseased\train2"/>
				<dirpath_in_testing_healthy dataType = "char" value = "#0\input\healthy\test"/>
				<dirpath_in_testing_diseased dataType = "char"  value = "#0\input\diseased\test"/>
				<dirpath_out_channelized_training_healthy dataType = "char" value = ""/>
				<dirpath_out_channelized_training_diseased dataType = "char" value = ""/>
				<dirpath_out_channelized_testing_healthy dataType = "char" value = ""/>
				<dirpath_out_channelized_testing_diseased dataType = "char" value = ""/>
				<filepath_out_training_tpf_fpf dataType = "char" value = "#0\output_temp\tpf_fpf\training\reader_02_training_tpf_fpf.txt"/>
				<filepath_out_training_auc_snr dataType = "char" value = "#0\output_temp\auc_snr\training\reader_02_training_auc_snr.txt"/>
				<filepath_out_testing_tpf_fpf dataType = "char" value = "#0\output_temp\tpf_fpf\testing\reader_02_testing_tpf_fpf.txt"/>
				<filepath_out_testing_auc_snr dataType = "char" value = "#0\output_temp\auc_snr\testing\reader_02_testing_auc_snr.txt"/>
				<filepath_out_reader_score_healthy_ref dataType = "char" value = "#0\output_temp\scores\reader_01_scores_healthy.txt"/>
				<filepath_out_reader_score_healthy_new dataType = "char" value = "#0\output_temp\scores\reader_02_scores_healthy.txt"/>
				<filepath_out_reader_score_diseased_ref dataType = "char" value = "#0\output_temp\scores\reader_01_scores_diseased.txt"/>
				<filepath_out_reader_score_diseased_new dataType = "char" value = "#0\output_temp\scores\reader_02_scores_diseased.txt"/>
			 </SingleSliceCHOModule>
			 <SingleSliceCHOModule>
				<reader_id dataType = "unsigned long" value = "3"/>
				<image_height dataType = "unsigned long" value = "64"/>
				<image_width dataType = "unsigned long" value = "64"/>
				<image_depth dataType = "unsigned long" value = "1"/>
				<one_image_file_per_frame dataType = "bool" value = "false"/>
				<target_slice_id dataType = "unsigned long" value = "1"/>
				<num_image_pairs_training dataType = "unsigned long" value = "16"/>
				<num_image_pairs_testing dataType = "unsigned long" value = "16"/>
				<channels_number dataType = "unsigned long" value = "15"/>	
				<channels_type dataType = "char" value = "LG"/>
				<channels_lg_spread dataType = "float" value = "22"/>
				<save_channelized_images dataType = "bool" value = "false"/>
				<auto_create_outputfolders dataType = "bool" value = "true"/>
				<dirpath_in_training_healthy dataType = "char" value = "#0\input\healthy\train3"/>
				<dirpath_in_training_diseased dataType = "char" value = "#0\input\diseased\train3"/>
				<dirpath_in_testing_healthy dataType = "char" value = "#0\input\healthy\test"/>
				<dirpath_in_testing_diseased dataType = "char"  value = "#0\input\diseased\test"/>
				<dirpath_out_channelized_training_healthy dataType = "char" value = ""/>
				<dirpath_out_channelized_training_diseased dataType = "char" value = ""/>
				<dirpath_out_channelized_testing_healthy dataType = "char" value = ""/>
				<dirpath_out_channelized_testing_diseased dataType = "char" value = ""/>
				<filepath_out_training_tpf_fpf dataType = "char" value = "#0\output_temp\tpf_fpf\training\reader_03_training_tpf_fpf.txt"/>
				<filepath_out_training_auc_snr dataType = "char" value = "#0\output_temp\auc_snr\training\reader_03_training_auc_snr.txt"/>
				<filepath_out_testing_tpf_fpf dataType = "char" value = "#0\output_temp\tpf_fpf\testing\reader_03_testing_tpf_fpf.txt"/>
				<filepath_out_testing_auc_snr dataType = "char" value = "#0\output_temp\auc_snr\testing\reader_03_testing_auc_snr.txt"/>
				<filepath_out_reader_score_healthy_ref dataType = "char" value = "#0\output_temp\scores\reader_02_scores_healthy.txt"/>
				<filepath_out_reader_score_healthy_new dataType = "char" value = "#0\output_temp\scores\reader_03_scores_healthy.txt"/>
				<filepath_out_reader_score_diseased_ref dataType = "char" value = "#0\output_temp\scores\reader_02_scores_diseased.txt"/>
				<filepath_out_reader_score_diseased_new dataType = "char" value = "#0\output_temp\scores\reader_03_scores_diseased.txt"/>
			 </SingleSliceCHOModule>
			 <SingleSliceCHOModule>
				<reader_id dataType = "unsigned long" value = "4"/>
				<image_height dataType = "unsigned long" value = "64"/>
				<image_width dataType = "unsigned long" value = "64"/>
				<image_depth dataType = "unsigned long" value = "1"/>
				<one_image_file_per_frame dataType = "bool" value = "false"/>
				<target_slice_id dataType = "unsigned long" value = "1"/>
				<num_image_pairs_training dataType = "unsigned long" value = "16"/>
				<num_image_pairs_testing dataType = "unsigned long" value = "16"/>
				<channels_number dataType = "unsigned long" value = "15"/>	
				<channels_type dataType = "char" value = "LG"/>
				<channels_lg_spread dataType = "float" value = "22"/>
				<save_channelized_images dataType = "bool" value = "false"/>
				<auto_create_outputfolders dataType = "bool" value = "true"/>
				<dirpath_in_training_healthy dataType = "char" value = "#0\input\healthy\train4"/>
				<dirpath_in_training_diseased dataType = "char" value = "#0\input\diseased\train4"/>
				<dirpath_in_testing_healthy dataType = "char" value = "#0\input\healthy\test"/>
				<dirpath_in_testing_diseased dataType = "char"  value = "#0\input\diseased\test"/>
				<dirpath_out_channelized_training_healthy dataType = "char" value = ""/>
				<dirpath_out_channelized_training_diseased dataType = "char" value = ""/>
				<dirpath_out_channelized_testing_healthy dataType = "char" value = ""/>
				<dirpath_out_channelized_testing_diseased dataType = "char" value = ""/>
				<filepath_out_training_tpf_fpf dataType = "char" value = "#0\output_temp\tpf_fpf\training\reader_04_training_tpf_fpf.txt"/>
				<filepath_out_training_auc_snr dataType = "char" value = "#0\output_temp\auc_snr\training\reader_04_training_auc_snr.txt"/>
				<filepath_out_testing_tpf_fpf dataType = "char" value = "#0\output_temp\tpf_fpf\testing\reader_04_testing_tpf_fpf.txt"/>
				<filepath_out_testing_auc_snr dataType = "char" value = "#0\output_temp\auc_snr\testing\reader_04_testing_auc_snr.txt"/>
				<filepath_out_reader_score_healthy_ref dataType = "char" value = "#0\output_temp\scores\reader_03_scores_healthy.txt"/>
				<filepath_out_reader_score_healthy_new dataType = "char" value = "#0\output_temp\scores\reader_04_scores_healthy.txt"/>
				<filepath_out_reader_score_diseased_ref dataType = "char" value = "#0\output_temp\scores\reader_03_scores_diseased.txt"/>
				<filepath_out_reader_score_diseased_new dataType = "char" value = "#0\output_temp\scores\reader_04_scores_diseased.txt"/>
			 </SingleSliceCHOModule>
			 <SingleSliceCHOModule>
				<reader_id dataType = "unsigned long" value = "5"/>
				<image_height dataType = "unsigned long" value = "64"/>
				<image_width dataType = "unsigned long" value = "64"/>
				<image_depth dataType = "unsigned long" value = "1"/>
				<one_image_file_per_frame dataType = "bool" value = "false"/>
				<target_slice_id dataType = "unsigned long" value = "1"/>
				<num_image_pairs_training dataType = "unsigned long" value = "16"/>
				<num_image_pairs_testing dataType = "unsigned long" value = "16"/>
				<channels_number dataType = "unsigned long" value = "15"/>	
				<channels_type dataType = "char" value = "LG"/>
				<channels_lg_spread dataType = "float" value = "22"/>
				<save_channelized_images dataType = "bool" value = "false"/>
				<auto_create_outputfolders dataType = "bool" value = "true"/>
				<dirpath_in_training_healthy dataType = "char" value = "#0\input\healthy\train5"/>
				<dirpath_in_training_diseased dataType = "char" value = "#0\input\diseased\train5"/>
				<dirpath_in_testing_healthy dataType = "char" value = "#0\input\healthy\test"/>
				<dirpath_in_testing_diseased dataType = "char"  value = "#0\input\diseased\test"/>
				<dirpath_out_channelized_training_healthy dataType = "char" value = ""/>
				<dirpath_out_channelized_training_diseased dataType = "char" value = ""/>
				<dirpath_out_channelized_testing_healthy dataType = "char" value = ""/>
				<dirpath_out_channelized_testing_diseased dataType = "char" value = ""/>
				<filepath_out_training_tpf_fpf dataType = "char" value = "#0\output_temp\tpf_fpf\training\reader_05_training_tpf_fpf.txt"/>
				<filepath_out_training_auc_snr dataType = "char" value = "#0\output_temp\auc_snr\training\reader_05_training_auc_snr.txt"/>
				<filepath_out_testing_tpf_fpf dataType = "char" value = "#0\output_temp\tpf_fpf\testing\reader_05_testing_tpf_fpf.txt"/>
				<filepath_out_testing_auc_snr dataType = "char" value = "#0\output_temp\auc_snr\testing\reader_05_testing_auc_snr.txt"/>
				<filepath_out_reader_score_healthy_ref dataType = "char" value = "#0\output_temp\scores\reader_04_scores_healthy.txt"/>
				<filepath_out_reader_score_healthy_new dataType = "char" value = "#0\output_temp\scores\reader_05_scores_healthy.txt"/>
				<filepath_out_reader_score_diseased_ref dataType = "char" value = "#0\output_temp\scores\reader_04_scores_diseased.txt"/>
				<filepath_out_reader_score_diseased_new dataType = "char" value = "#0\output_temp\scores\reader_05_scores_diseased.txt"/>
			 </SingleSliceCHOModule>
			<MRMCModule>
				<file_path_scores_diseased dataType = "char" value = "#0\output_temp\scores\reader_05_scores_diseased.txt"/>
				<file_path_scores_healthy dataType = "char" value = "#0\output_temp\scores\reader_05_scores_healthy.txt"/>
				<file_path_results dataType = "char" value = "#0\outputTest\shot1result_LG_multi.txt"/>
				<flag_estimate_biased dataType = "int" value = "0"/>
				<flag_verbose_required dataType = "int" value = "1"/>
			</MRMCModule>
		</REF>	
	</TEMPLATE_MEVIC_SIMULATION>
</SuperXML>
\end{lstlisting}

\subsection{Command}

\lstset{language=C++}
\begin{lstlisting}
run_simulation.bat
\end{lstlisting}

In this file, the data are reorganized with batch commands for training and testing the different observers with 17 pairs of images.

ThenIn this command the VCT.exe is executed for training and testing the observers and running the MRMC analysis.

\subsection{output}

Several files are genrated the same outcome than in the previous section for each observer and then one file for the final results:

\begin{itemize}
\item shot1result\_LG\_multi.txt
\end{itemize}

\subsection{Module: "MRMCModule"}

Modules for running the multi readers multi cases statistical analysis.

\subsubsection{Parameter: "file\_path\_scores\_diseased"}

The file path with the scores for the diseased images.

\subsubsection{Parameter: "file\_path\_scores\_healthy"}

The file path with the scores for the healthy images.

\subsubsection{Parameter: ""file\_path\_results}

This is the file path of the final results.


