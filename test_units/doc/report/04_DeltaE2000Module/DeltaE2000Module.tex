\chapter{$\Delta{E_{2000}}$ display simulation}
\minitoc{}

This simulation compares two colored $XYZ$ images or sequences and computes the $\Delta{E_{2000}}$.

\section{Introduction}

The simulation loads two previously generated simulations.

\subsection{1$^{st}$ input}

This xml is used for generating the outcome of a monitor on which a gradient is displayed (cf Figure \ref{fig:gradientRGBdeltaE}). Except the last module the other modules have been described in the previous chapters.

\lstset{language=XML}
\begin{lstlisting}
<?xml version="1.0" encoding="ISO-8859-1"?>
<SuperXML>
	<LIST_OF_ITERATIONS name = "pipeline" value="1">	
	</LIST_OF_ITERATIONS>
	
	<LIST_OF_PARAMETERS>
		<LocalParameter name = "REF">
			<SequenceRawGeneratorModule name="directory" value = "#0\input\"/>
			<Apply3x1DLutModule name="filenamePathLutRed" value = "#0\input\lut_red.txt"/>
			<Apply3x1DLutModule name="filenamePathLutGreen" value = "#0\input\lut_green.txt"/>
			<Apply3x1DLutModule name="filenamePathLutBlue" value = "#0\input\lut_blue.txt"/>
		</LocalParameter>				
	</LIST_OF_PARAMETERS>

	<TEMPLATE_MEVIC_SIMULATION>
		<REF>
			<SequenceRawGeneratorModule>
				<directory dataType="char" value = "$"/>
				<number_of_slices dataType="int" value = "1"/>
				<width dataType = "int" value = "64"/>
				<height dataType = "int" value = "64"/>
				<nbBitsRange dataType = "int" value = "8"/>
				<_0bigEndian_1littleEndian dataType = "int" value = "0"/>
				<nbBitsOutput dataType = "int" value = "8"/>
				<nbBitsPrecision dataType = "int" value = "24"/>
				<_1WhiteIs0_0Otherwise dataType = "int" value = "0"/>
				<frame_repeat dataType = "int" value = "1"/>
				<_1RGB_0GRAY dataType = "int" value = "1"/>			
			</SequenceRawGeneratorModule>
			<Apply3x1DLutModule>
				<filenamePathLutRed dataType="string" value="$"/>
				<filenamePathLutGreen dataType="string" value="$"/>
				<filenamePathLutBlue dataType="string" value="$"/>
				<nbBits dataType="unsigned int" value="8"/>
			</Apply3x1DLutModule>	
			<SRgbDisplayModule>
				<gamma dataType="float" value="2.2"/>
				<lumMax dataType="float" value="250.0"/>
				<contrast dataType="unsigned int" value="750"/>
				<illum dataType="string" value="D50"/>
			</SRgbDisplayModule>			
			<SaveFrameTXTModule>
				<Filename dataType="char" value="#0\output\RGBINREF"/>
				<ComponentToWrite dataType="int" value="0"/>
				<FrameToWrite dataType="int" value="-1"/>
				<ChannelToWrite dataType="int" value="-1"/>
			</SaveFrameTXTModule>
			<SaveFrameTXTModule>
				<Filename dataType="char" value="#0\output\XYZOUTREF"/>
				<ComponentToWrite dataType="int" value="2"/>
				<FrameToWrite dataType="int" value="-1"/>
				<ChannelToWrite dataType="int" value="-1"/>
			</SaveFrameTXTModule>	
			<WriterModule>
				<Filename dataType="char" value="#0\output\XYZOUTREF.bin"/>
				<compression dataType="bool" value="1"/>
				<savelastcomponent dataType="bool" value="1"/>
			</WriterModule>			
		</REF>	
	</TEMPLATE_MEVIC_SIMULATION>
</SuperXML>
\end{lstlisting}

The input image is a grayscale gradient and is shown in the figure \ref{fig:gradientRGB}.

This sub simulation generates a bin file that is the input of to the $\Delta{E_{2000}}$ simulation.

\begin{figure}[!htb]
\begin{center}
\includegraphics[width=0.5\columnwidth]{./04_DeltaE2000Module/images/24RGBL.png}
\caption{Input grayscale gradient image.}\label{fig:gradientRGBdeltaE}
\end{center}
\end{figure}

\subsubsection{WriterModule}

This module is used for saving the simulation in a binary file called container file and its extension is ".bin".

\paragraph{Filename}

Input filename and path of the output file.

\paragraph{compression}

This parameter should be equal to "0" or "1". "1" means that the data are compressed.

\paragraph{savelastcomponent}

This parameter should be equal to "0" or "1". "1" means that only the last component of the simulation is saved.

\subsection{2$^{nd}$ input}

This xml is used for generated the outcome of a monitor on which a noisy gradient is display (cf Figure \ref{gradientNoisyRGBdeltaE}). Except the last module the other modules have been described in the previous chapters.

\lstset{language=XML}
\begin{lstlisting}
<?xml version="1.0" encoding="ISO-8859-1"?>
<SuperXML>
	<LIST_OF_ITERATIONS name = "pipeline" value="1">	
	</LIST_OF_ITERATIONS>
	
	<LIST_OF_PARAMETERS>
		<LocalParameter name = "REF">
			<SequenceRawGeneratorModule name="directory" value = "#0\input\"/>
			<Apply3x1DLutModule name="filenamePathLutRed" value = "#0\input\lut_red.txt"/>
			<Apply3x1DLutModule name="filenamePathLutGreen" value = "#0\input\lut_green.txt"/>
			<Apply3x1DLutModule name="filenamePathLutBlue" value = "#0\input\lut_blue.txt"/>
		</LocalParameter>				
	</LIST_OF_PARAMETERS>

	<TEMPLATE_MEVIC_SIMULATION>
		<REF>
			<SequenceRawGeneratorModule>
				<directory dataType="char" value = "$"/>
				<number_of_slices dataType="int" value = "1"/>
				<width dataType = "int" value = "64"/>
				<height dataType = "int" value = "64"/>
				<nbBitsRange dataType = "int" value = "8"/>
				<_0bigEndian_1littleEndian dataType = "int" value = "0"/>
				<nbBitsOutput dataType = "int" value = "8"/>
				<nbBitsPrecision dataType = "int" value = "24"/>
				<_1WhiteIs0_0Otherwise dataType = "int" value = "0"/>
				<frame_repeat dataType = "int" value = "1"/>
				<_1RGB_0GRAY dataType = "int" value = "1"/>			
			</SequenceRawGeneratorModule>
			<Apply3x1DLutModule>
				<filenamePathLutRed dataType="string" value="$"/>
				<filenamePathLutGreen dataType="string" value="$"/>
				<filenamePathLutBlue dataType="string" value="$"/>
				<nbBits dataType="unsigned int" value="8"/>
			</Apply3x1DLutModule>	
			<SRgbDisplayModule>
				<gamma dataType="float" value="2.2"/>
				<lumMax dataType="float" value="250.0"/>
				<contrast dataType="unsigned int" value="750"/>
				<illum dataType="string" value="D50"/>
			</SRgbDisplayModule>			
			<SaveFrameTXTModule>
				<Filename dataType="char" value="#0\output\RGBINTEST"/>
				<ComponentToWrite dataType="int" value="0"/>
				<FrameToWrite dataType="int" value="-1"/>
				<ChannelToWrite dataType="int" value="-1"/>
			</SaveFrameTXTModule>
			<SaveFrameTXTModule>
				<Filename dataType="char" value="#0\output\XYZOUTTEST"/>
				<ComponentToWrite dataType="int" value="2"/>
				<FrameToWrite dataType="int" value="-1"/>
				<ChannelToWrite dataType="int" value="-1"/>
			</SaveFrameTXTModule>	
			<WriterModule>
				<Filename dataType="char" value="#0\output\XYZOUTTEST.bin"/>
				<compression dataType="bool" value="1"/>
				<savelastcomponent dataType="bool" value="1"/>
			</WriterModule>			
		</REF>	
	</TEMPLATE_MEVIC_SIMULATION>
</SuperXML>
\end{lstlisting}

\begin{figure}[!htb]
\begin{center}
\includegraphics[width=0.5\columnwidth]{./04_DeltaE2000Module/images/24RGBLNoise.png}
\caption{Input grayscale gradient noisy image.}\label{fig:gradientRGBdeltaENoisy}
\end{center}
\end{figure}

The input image is a grayscale gradient and is shown in the figure \ref{fig:gradientRGBdeltaENoisy}.

This sub simulation generates a bin file with the simulation to give to the deltaE2000 simulation.

\subsubsection{WriterModule}

This module is used for saving the simulation in a binary file called container file and its extension is ".bin".

\paragraph{Filename}

Input filename and path of the output file.

\paragraph{compression}

This parameter should be equal to "0" or "1". "1" means that the data are compressed.

\paragraph{savelastcomponent}

This parameter should be equal to "0" or "1". "1" means that only the last component of the simulation is saved.

\section{Command}

\lstset{language=C++}
\begin{lstlisting}
"..\vct\VCT.exe" SuperXML ".\xml\test_DeltaE2000Module.xml" "."
\end{lstlisting}

The first argument is the path to the VCT executable. The second argument is simply the key word "SuperXML" and the third one is the path to the xml file. Any additional parameters are used within the xml file, in this case, it corresponds to an input path.

\section{xml}

The input xml file is:

\lstset{language=XML}
\begin{lstlisting}
<?xml version="1.0" encoding="ISO-8859-1"?>
<SuperXML>
	<LIST_OF_ITERATIONS name = "pipeline" value="1">	
	</LIST_OF_ITERATIONS>
	<LIST_OF_PARAMETERS>
		<LocalParameter name = "REF">
		</LocalParameter>				
	</LIST_OF_PARAMETERS>
	<TEMPLATE_MEVIC_SIMULATION>
		<REF>
			<ReaderModule name="ReaderModule">
				<Filename dataType="string" value="#0\input_ref\output\XYZOUTREF.bin" />
				<Compression dataType="bool" value="1" />
			</ReaderModule>
			<ReaderModule name="ReaderModule 2">
				<Filename dataType="string" value="#0\input_test\output\XYZOUTTEST.bin" />
				<Compression dataType="bool" value="1" />
			</ReaderModule>	
			<DeltaE2000Module name="DeltaE2000Module">
			</DeltaE2000Module>	
			<SaveFrameTXTModule>
				<Filename dataType="char" value="#0\output\dE2000"/>
				<ComponentToWrite dataType="int" value="0"/>
				<FrameToWrite dataType="int" value="-1"/>
				<ChannelToWrite dataType="int" value="-1"/>
			</SaveFrameTXTModule>			
		</REF>	
	</TEMPLATE_MEVIC_SIMULATION>
</SuperXML>
\end{lstlisting}

\section{Output}

The output is a an image or a sequence or frames containing the metric value $\Delta{E_{2000}}$, pixel per pixel comparison between the first and the second input (cf Figure \ref{fig::mapDeltaE2000}). The minimum $\Delta{E_{2000}}$ value is $0.5$ and the maximum value is $36.51$.

\begin{figure}[!htb]
\begin{center}
\includegraphics[width=0.5\columnwidth]{./04_DeltaE2000Module/images/deltaE.png}
\caption{Visualization of the map of delta E 2000 differences in grayscale between image from Figure \ref{fig:gradientRGBdeltaE} and Figure \ref{fig:gradientRGBdeltaENoisy}. The minimum $\Delta{E_{2000}}$ value is $0.5$ and the maximum value is $36.51$. The perception threshold is 1.}\label{fig:mapDeltaE2000}
\end{center}
\end{figure}

\section{Module: ReaderModule}

\subsection{Parameter: "Filename"}

Input filename and path of the input ".bin" file.

\paragraph{compression}

This parameter should be equal to "0" or "1". "1" means that the data are compressed.

\section{Module: DeltaE2000Module}

This module does not have any input parameter, it computes the $\Delta{E_{2000}}$ as defined by the $CIE$ from two input containers.